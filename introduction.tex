\chapter{\label{cha:intro}Introduction}

\listoftodos{}

As software systems continue to grow in size and complexity, the occurrence of software bugs becomes an inevitable challenge. Industry experience shows that software often contains 1-25 bugs per thousand lines of code\cite{code-complete} and the number of bugs increases quadratically with the size of the codebase\cite{month}. Software bugs can lead to critical program errors or even crashes, which, in turn, may result in significant financial losses\cite{bug4} or pose serious risks to safety and life\cite{bug1, bug2, bug3}. To address the threat posed by bugs or vulnerability of programs, researchers have investigated a variety of bug detection techniques. 

There are two major ways to detect bugs: formal verification methods and testing methods. Formal verification techniques, such as axiomatic approaches, use mathematical deduction to prove the absence of bugs. These approaches heavily rely on the expertise of developers and normally require a significant amount of time and effort\cite{sel4}. Formal methods can sometimes produce false positives, which further increases the complexity and time cost of verification.
Automated testing, on the other hand, has for long been of great importance for its scalability and efficiency. Although testing cannot prove full correctness of a program, it is very effective in finding real bugs. Various testing methods have been developed over the years, including static analysis\cite{infer, RacerD} and dynamic testing tools\cite{ASAN, TSAN}. However, due to the complexity of the programs being tested, static analysis tools do not always report bugs comprehensively or correctly. Dynamic testing, on the other hand, often requires high-quality test cases to cover a large portion of program behaviors, which may demand a deep understanding of the program under test and can be time-consuming and require significant effort to complete.


% Testing techniques aimed at discovering bugs and safety vulnerabilities have been developed over the years, primarily including static analysis and dynamic testing. Static analysis tools\cite{infer, RacerD} typically perform bug detection at compile time, relying on some abstraction of the semantics of the source code. 
% Due to the complexity of the programming language and the logic of the tested program, such static analysis tools sometimes do not ensure reporting bugs comprehensively and correctly. The other way is dynamic testing, which explores reachable states at runtime. Unit testing, for example, plays a fundamental role in software development. Some influential tools have been developed, including AddressSanitizer\cite{ASAN}, detecting addressability issues, and ThreadSanitizer\cite{TSAN}, which detects data races and deadlocks. However, even though high quality test cases can cover large ranges of program behaviors, there can still be some corner cases that are not exposed. In addition, such test cases require a deep understanding of the program under test and consume a considerable amount of time and effort to complete. 

% \michalis{The paragraph above is all over the place, and doesn't motivate fuzzing well IMO. I would suggest a structure like: \\
%  - bugs can be catastrophic, bug detection is important\\
%  - two ways to detect bugs: (1) formal methods establish their absence, but either don't scale or have false positives (2) testing cannot prove full correctness, but is very effective in finding real bugs\\
%  - testing, however, can be inadequate (why?)\\
% ~\\
%  In the next paragraph you can introduce fuzzing by explaining how it overcomes certain shortcomings of testing. In principle, you could even avoid talking about formal verification and focus on bug detection.
% }

Fuzz testing has become increasingly popular in recent years. It repeatedly executes the tested program by generating random inputs and monitors for any observed buggy behaviors. These approaches are usually easy to apply and scale well. Fuzzers can be classified into three categories based on the level of knowledge they use from the tested programs: black-box, grey-box and white-box fuzzers. A fuzzer typically has a feedback loop. It maintains a set of seeds as program inputs to execute the program. The information about the execution is collected to determine whether a seed is interesting, which means the seed has triggered new interested behaviors, such as covering new code branches. The interesting seeds will be used for generating new seeds during repeated executions. One of the most popular fuzzing tool is AFL\cite{afl}, a coverage-guided mutation-based grey-box fuzzer which achieves superior efficiency than previous black-box fuzzers. Since then, researchers have developed various techniques to improve the code coverage and accelerate the bug detection. 

% \michalis{The different fuzzing approaches (blackbox, greybox etc) are not explained. Similarly for coverage-guided fuzzing. After motivating it, maybe introduce it by explaining how it works on a high level (you do explain some things below, but not the terms above).}

% A fuzzer typically has a feedback loop. It maintains a set of seeds as program inputs to execute the program. The information about the execution is collected to determine whether a seed is interesting, which means the seed has triggered new interested behaviors. The interesting seeds will be used for generating new seeds during repeated executions. A mutation-based fuzzer can mutate (such as bit flipping, hashing, shifting, etc) the interesting seeds to generate new seeds. 



Entering the multi-core era, concurrent programming has gained increasing significance. The need for testing concurrent programs has also grown considerably. Researchers have developed various testing techniques, and fuzzing for concurrent programs has gained increasing attention. Traditional coverage-guided approaches face challenges in detecting concurrency bugs, as code coverage information does not reflect thread interleavings, which can lead to such bugs. Therefore, thread-relevant instrumentation is needed to provide concurrency feedback information in the fuzzing loop. Another problem is that, assuming sequential consistency, both the program input and the thread interleavings (or schedules) determine the program's behavior. Hence existing concurrency fuzzers can be classified into two types: fuzzers aiming for generating seeds\cite{muzz} and thread interleavings\cite{rff, conzzer}.

However, current concurrency fuzzers mainly focus on testing for programs under sequential consistency memory model. Modern computer architectures, such as Power\cite{Power} and ARM\cite{arm, ARMv8}, often allow speculative and out-of-order executions and introduce cache hierarchies to reduce memory access latency. On the one hand, although sequential consistency is easy to understand by programmers, achieving it is very expensive. On the other hand, by relaxing the memory order and allowing for weak memory behaviors, the efficiency of execution can be significantly improved. However, both developing and testing programs under weak memory has been notoriously hard. Given the success that fuzzing has achieved on sequential programs and multi-threading programs under SC memory, it is reasonable to believe fuzzing can also be helpful for weak memory testing. 

% One commonly used method for testing concurrent programs is controlled concurrency testing (CCT). CCT repeatedly executes the program with probabilistic guarantees or proactively controls thread interleavings based on specific schedules. For example, PCT\cite{pct} limits the number of thread switches, characterized by bug depth, and assigns random change points to create different schedules. However, CCT does not use program feedback to generate test cases and sometimes relies on prior knowledge of the program under test, such as PCT.
% On the other hand, 

% \michalis{Is CCT relevant to the discussion, or would it suffice to say that fuzzing does not work for concurrency and explain why?}

% Most concurrency fuzzers mainly focus on testing for programs under sequential consistency memory model. However, modern computer architectures often allow for weak memory behaviors. On the one hand, although SC is easy to understand by programmers, achieving SC is very expensive. On the other hand, by relaxing the memory order and allowing for weak memory behaviors, the efficiency of execution can be significantly improved.
%  However, both developing and testing programs under weak memory has been notoriously hard. Given the success that fuzzing has achieved on sequential programs and multi-threading programs under SC memory, it is reasonable to believe fuzzing can also be helpful for weak memory testing. 

% \michalis{If the main point of the introduction is that we now support weak memory, you can center the introduction around that (and not only mention it at the very end). E.g.,: (1) bugs are bad; there is testing+fuzzing (2) testing does not work for concurrency, and even if it does it just for SC (3) most platforms employ weak memory (explain what this means) and there is no testing technique for weak memory. You don't have to go into details of different techniques (sanitizers, AFL, CCTs etc), even citations would suffice (you can do that in the related work section)}

In this thesis, we propose a novel fuzzing approach designed to support weak memory models. Unlike existing fuzzers that rely on random seeds or thread schedules, our approach mutates execution graphs to generate test cases. Due to the generality of execution graphs, our fuzzer also supports programs under the sequential consistency memory model. We then present two implementations in both C11Tester and GenMC, which are state-of-the-art platforms for testing weak memory programs.

% \michalis{What are execution graphs?} % added at "Chapter~\ref{cha:background} provides the ... "

The rest of this thesis is structured as follows: Chapter~\ref{cha:background} provides the background information on fuzzers, weak memory models, execution graphs and the C/C++11 memory model. Chapter~\ref{cha:fuzz} presents the intuition and a high level overview of the fuzzing algorithm. Chapter~\ref{cha:c11tester} describes the implementation on C11Tester, with the evaluation results and discussion. Chapter~\ref{cha:genmc} describes the implementation on GenMC and the evaluation of three mutation strategies. Chapter~\ref{cha:related} briefly summarizes some other related work on fuzzing, memory models and model checking, etc. Chapter~\ref{cha:conclusion} concludes this thesis and discusses possible future work. 






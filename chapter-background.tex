\chapter{\label{cha:title}Background}

Short chapter intro \ldots

\section{Weak Memory Models}

In concurrent programming, shared memory is used for sharing data and passing messages among threads. Memory models are essential for programmers to reason about their code, and for compiler and hardware manufacturers to implement low-level supporting infrastructures. The simplest memory model, proposed by Lamport\cite{SC} in 1979, is the Sequential Consistency Model (SC) . Under the SC model, intra-thread instructions are executed following their program order and threads can interleave in any order. A read operation can only read from the most recent value written to the same memory location. The SC is also known as the strong memory model, with other non-SC memory models referred to as weak memory models.

Consider the store buffer (SB) \ref{SB} example, where x, y are shared variables, and r1, r2 are local variables, all initialized with 0. It can be seen that under SC, none of the possible thread interleavings (e.g., abcd, acbd, acdb, ...) results in both r1 and r2 reading the value 1.

\lstset{ %
  language=C++,               % set the language to C++
  basicstyle=\ttfamily\small, % the size of the fonts that are used for the codeline-numbers
  stepnumber=1,               % the step between two line-numbers. If it's 1, each line will be numbered
  numbersep=5pt,              % how far the line-numbers are from the code
  backgroundcolor=\color{white}, % choose the background color. You must add \usepackage{xcolor}
  showspaces=false,           % show spaces adding particular underscores
  showstringspaces=false,     % underline spaces within strings
  showtabs=false,             % show tabs within strings adding particular underscores
  frame=none,               % adds a frame around the code
  rulecolor=\color{black},    % if not set, the frame-color may be changed on line-breaks within not-black text (e.g. comments (green here))
  tabsize=2,                  % sets default tabsize to 2 spaces
  captionpos=b,               % sets the caption-position to bottom
  breaklines=true,            % sets automatic line breaking
  breakatwhitespace=false,    % sets if automatic breaks should only happen at whitespace
  keywordstyle=\color{blue},  % keyword style
  commentstyle=\color[rgb]{0.2,1,0.4},  % comment style
  stringstyle=\color{red},    % string literal style
  xleftmargin=40pt,           % left margin for the whole code block
  xrightmargin=40pt           % right margin for the whole code block
}

\begin{lstlisting}[caption={SB}, label={SB}]

x = 0;
y = 0;
void thread1() {
    x  = 1;  // (a) 
    r1 = y;  // (b)
}
void thread2() {
    y  = 1;  // (c)
    r2 = x;  // (d)
}

\end{lstlisting}

% x = 0;
% y = 0;
% // thread 1:
% a: x = 1;
% b: r1 = y;
% // thread 2
% c: y = 1;
% d: r2 = x;

However, this behavior can be allowed by some weak memory models provided by hardware architectures and programming languages. Take TSO (total store order)\cite{TSO}, supported by x86 architectures, for example. In TSO model, each thread has a local store buffer. Values written to shared memory will be first stored in the buffer and some time in the future, will be flushed to the shared memory. The store buffer has the FIFO property, hence the ordering of all writes in the same thread will not be broken. 

For the SB example, if the momery model is TSO, it is possible that after executing assignments a and b, the values are buffered, followed by r1 and r2 reading 0, and finally the buffered values updated to the shared memory. 

Some weak memory behaviors can be forbidden by one weak memory model, but allowed by another. In the following message passing (MP) \ref{MP} example, after data is set to 1, the sender thread initialize the pointer, p, with the address of data, hoping the receiver thread only use the data after the pointer is initialized (inidicating data is set). Under TSO, because of the FIFO property of store buffers, the shared variable p is initialized only after the updating of data is finished. But this is not guaranteed under the PSO (partial store order) model\cite{PSO}. In PSO, each memory location has a seperate FIFO store buffer in a thread. In this case, the ordering of moving the values of data and p from their buffers to the shared momery is not restricted. The receiver thread can read y=1 when data has not been updated yet. 

\begin{lstlisting}[caption={MP}, label={MP}]
p = nullptr;
data = 1;
// sender thread
void sender() {
    data = 1;
    p = &data;
}
// receiver thread
void receiver() {
    while(p == nullptr) {;}
    use(*p);
}
\end{lstlisting}
% p = nullptr
% data = 0
% // sender
% data = 1;
% p = &data;
% // receiver 
% while(p == nullptr) {;}
% use(*p)

There are a variety of other weak memory models, such as the ARMv8, supporting out-of-order executions and speculative executions, and language level memory models, including the JAVA memory model\cite{java} and C++ memory model. The rest of this paper mainly discuss the C/C++ memory model\cite{c++model}. 



\section{C/C++ Memory Model}
C/C++ provides additional concurrency primitives, including atomics, mutex, threads and fences, along with a extensive specification of its memory model.
The first C/C++ memory model was described in a proposal\cite{c++model-proposal} in 2008, which was refined and formalized by \cite{c++model}. The following contents use the notations and definitions in \cite{c++model}, unless otherwise specified. 

The memory model can be defined as a function, taking a set of candidate executions $X$ as input. These executions must be allowed by the operational semantics and are consistent, denoted as pre-executions. The function returns "NONE" if any executions have undefined behaviors; otherwise, it returns "SOME" pre-executions.

A candidate execution X contains two components, $X = (X_{opsem}, X_{witness})$, where $X_{opsem}$ is determined by the operational semantics and $X_{witness}$ is an existential witness of some further data, both are composed of some memory actions (actions for short) and relations. An execution can be represented as a graph, with its actions as nodes and relations as edges. An action can be a non-atomic read or write, atomic operations, mutex operations and fences, represented by <aid, tid, type, location, value>. The $X_{opsem}$ contains three type of relations: 

\begin{itemize}
    \item \textit{sequenced-before} (sb): A relation between intra-thread actions given by C/C++ language specifications, usually analogous to program order. When two seperate actions are written in two seperate statements, the former is sequenced-before the latter. Arguments of functions or operands of some operators like '==' do not have specified evaluation order, thus do not have sequenced-before relations. 
    \item \textit{additional-synchronized-with} (asw): The thread-creation action introduces an asw relation from the sequenced-before-maximal actions of the parent thread to the sequenced-before-minimal actions of the child.
    \item \textit{data-dependency} (dd):  The dd is provided by the operational semantics, primarily used for release/consume atomics. For example, a store to a pointer and the use of the pointed data have a dd relation. 
\end{itemize}

In the SB example, assuming x and y atomic variables, the $X_{opsem}$ of a candidate execution can be drawn as: 
%   x = 0
%     | sb
%   y = 0
%   /asw    \asw
% x = 1     y = 1
%   |sb       |sb
% read y    read x

The $X_{witness}$ part contains additional three relations. These relations are not uniquely determined by the operational semantics. Therefore, given a program p, the candidate execution X can only have one $X_{opsem}$, but have multiple choices of $X_{witness}$. 

\begin{itemize}
    \item \textit{read-from} (rf): If a read action (non atomic read, atomic read, rmw) reads a value from a write action (non atomic write, atomic write, rmw), an rf edge from the write to the read is established. In addition, a lock and its last preceding unlock action of the same mutex also establish an rf. The rf reads-from map is defined as a function containing all these rf relations in the execution. 
    \item \textit{modification-order} (mo): A total order of all writes to the same atomic location. Each location can have an independent mo "chain", unrelated to other locations.
    \item \textit{sequentially-consistent} (sc): Totally orders all mutex actions and actions with \texttt{memory\_order\_seq\_cst} memory order.
\end{itemize}

In the SB example, assuming the initializations are non-atomic and other writes and reads are \texttt{memory\_order\_seq\_cst}, a possible $X_{witness}$ for the SB example can be:
% rf
% x=1 -rf-> read x in thread2
% y=1 -rf-> read y in thread1
% sc
% x=1 -sc-> y=1 -sc-> read y -sc-> read x
% mo
% y=0 -mo-> y=1
% x=0 -mo-> x=1

There are some derived relations defined based on the above six relations. These derived relations will help to define the memory model and rule out illegal executions.

\begin{itemize}
    \item \textit{synchronizes-with} (sw): Every unlock action of a mutex has an sw edge pointing to the lock odered after it in the sc order mentioned above. All asw relations are sw. A read-acquire (read with \texttt{memory\_order\_acquire}) reading from a write-release gives rise to a sw relation. More generally, when the read-acquire R reads from a write W, it also sw other write-release that is ordered before W in the modification order. However, not all write-releases preceding W can have sw relations with W, only those contained by the \textit{release sequence} of W. The definition of \textit{release sequence} is omitted here. 
    \item \textit{dependency-ordered-before} (dob): Similar to sw in release/acquire pairs, dob is introduced for release/consume pairs. The formal definition is ommited here. Instead, take the MP example for illustration. The reading and dereferencing of p carry a dd relation given by the operational semantics, which forms a \textit{carries-a-dependency-to} (cad) relation. When \texttt{p==nullptr} reads from \texttt{p=\&data}, they from a \textit{dependency-ordered-before} relation. As a result, \texttt{*p}, having a cad with \texttt{p==nullptr}, also has a dob from \texttt{p=\&data}. 
    \item \textit{happens-before} (hb): If the execution has no consume operations, hence no dob relations, the hb relation is a transitive closure of $sb \cup sw$. More generally, hb is defined as the union of sb and \textit{inter-thread-happens-before}, which combines the sw and dob relations.
\end{itemize}


The three relations in $X_{witness}$ (rf, mo and sc) can not be arbitrarily composed to make an execution. Instead, they have to satisfy some constraints, called \textit{coherence}. The coherence constraints have the form "A-B Coherence", or CoAB, where both A and B can be a read or write and  $A \xrightarrow{\text{hb}} B$. As illustrated before, the hb is derived from sb, sw and dob, where sw is derived from sc and rf and dob is derived from rf. The constraints on hb, mo and rf will ultimately constain the combinations of rf, mo and sc. The coherence constraints are listed below:

\begin{itemize}
    \item \textit{Read-Read Coherence} (CoRR): Two reads ordered by hb cannot read from two writes ordered by mo in the other direction. 
    \item \textit{Write-Read Coherence} (CoWR): When a write, $w$, happens before a read $r$, $r$ cannot read from a write that precedes $w$ in mo. 
    \item \textit{Write-Write Coherence} (CoWW): The mo and hb relations of two writes, $w_1$ and $w_2$, should have same directions. For instance, $w_1 \xrightarrow{\text{mo}} w_2 \land w_2 \xrightarrow{\text{hb}} w_1$ is not allowed. 
    \item \textit{Read-Write Coherence} (CoRW): When a read happens before a write $w$, it cannot read from a write that is ordered after $w$ in mo. This forbids the $rf \cup hb \cup mo$ to be cyclic.
\end{itemize}


Take the SB example, if the reads and writes are SC atomic operations, the following execution where both reads read the value 0 is not allowed, since this execution violates the CoWR constraint. However, both reads reading the value 0 from the initializations is allowed when the reads and writes are relaxed. 
% rf
% x=0 -rf-> read x in thread2
% y=0 -rf-> read y in thread1
% sc
% x=1 -sc-> y=1 -sc-> read y -sc-> read x
% mo
% y=0 -mo-> y=1
% x=0 -mo-> x=1
% \begin{figure}
  %\footnotesize
  \centering
  \begin{tikzpicture}[yscale=0.85]
    \node (s) at (0,0)  {\begin{tabular}{c}$i: [X=Y=0]$ \\ {\footnotesize\color{teal}$\set{(X,i_x),(Y,i_y)}$}\end{tabular}};
    \node (t11) at (-3,-2)  {\begin{tabular}{c}$e_1: \W(X,1)$ \\ {\footnotesize\color{teal}$\set{(X,e_1),(Y,i_y)}$}\end{tabular}}; 
    \node (t21) at (0,-2)  {\begin{tabular}{c}$e_2: \R(X, 1)$  \\ {\footnotesize\color{teal}$\set{(X,e_1),(Y,i_y)}$}\end{tabular}};
    \node (t22) at (0,-4) {\begin{tabular}{c}$e_3: \W(Y,1)$ \\ {\footnotesize\color{teal}$\set{(X,e_1),(Y,e_3)}$}\end{tabular}};
    \node (t31) at (3, -2)  {\begin{tabular}{c}$e_4: \R(Y, 1)$ \\ {\footnotesize\color{teal}$\set{(X,i_x),(Y,e_3)}$}\end{tabular}};
    \node (t32) at (3, -4) {\begin{tabular}{c}$e_5: \R(X, 0)$ \\ {\footnotesize\color{teal}$\set{(X,i_x),(Y,e_3)}$}\end{tabular}};
  
  % \node at (1.2,-1.75)  {{\footnotesize\color{blue}$\set{(X,e_1)}$}};
  % \node at (3.7,-1.75)  {{\footnotesize\color{blue}$\set{(Y,e_3)}$}};
  
  %
    \draw[po] (s) to node[right]{(1)} (t11);
    \draw[po] (s) to node[right]{(2)} (t21) ;
    \draw[po] (t21) to node[right]{(3)} (t22);
    \draw[po] (s) to node[right]{(4)} (t31);
    \draw[po] (t31) to node[right]{(5)} (t32);
    \draw[rf] (t11) to node[above]{(2)} node[below]{$\rf$} (t21) ;
    \draw[rf] (t22) to node[above]{(4)} node[below]{$\rf$} (t31) ;
  \end{tikzpicture}
  %\caption{A $d=2$ execution of Program~\ref{prg:MP2} that hits the assertion violation}
  \caption{A $d=2, \; h=1$ execution of MP2. The illustrated execution selects $[e_2, e_4]$ as the sinks of the two communication relations and assigns initial priorities as $[T1, T2, T3]$.}
  \label{fig:mp2:d2}
  \end{figure}

The above mentioned C/C++ memory model is an axiomatic model, specifying which executions are allowed and which are not. However, this model is deemed to be flawed. Firstly, the data-dependency relation and those associated with consume-release atomics are specified but are not implemented on most platforms, where they are typically treated as acquire-release atomic operations. Furthermore, there exist some executions actually allowed by the model but should be rulled out in principle.  For instance, in a load buffer (LB) program, where two threads load from different locations and store the loaded value to the other location. If all the atomic operations have a relaxed memory order, any loaded value is allowed by the model. The problem is called \textit{out-of-thin-air} (OOTA) problem. Since the model was proposed, enumerous efforts have been taken to revise it. These will be summarized in section \ref{cha:related}.

% execution graph

% is rf sufficient for execution

Fuzzing has been a popular approach in the domain of software testing due to its efficiency and capability to uncover  unexpected bugs. Fuzz testing was originally developed in the days of sequential programs. With the rise of multi-core devices and increasing demand for computational efficiency, the prevalence of concurrent programming has led to a new wave of research applying fuzz testing techniques. In recent years, several fuzzers have been proposed for sequentially consistent multi-threading programs, a subset of concurrent programs, using thread interleaving semantics. However, the exploration of fuzzing techniques for weak memory concurrency remains limited.

This thesis presents a novel fuzzing approach for programs under weak memory models. It targets at the execution graphs, instead of schedules, and performs graph mutations to guide new executions.

We implement the fuzzer based on two state-of-the-art testing tools: C11Tester and GenMC. Different mutation strategies are explored for comparison. Benchmark results demonstrate that our fuzzer explores a broader range of execution graphs compared to random testing, resulting in improved bug detection.

\textbf{Key Words:} fuzz testing, weak memory model, execution graph, concurrency bugs

